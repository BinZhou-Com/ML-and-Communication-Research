\documentclass[12pt,a4paper]{report}
\newcommand{\dd}[1]{\mathrm{d}#1}
\usepackage[utf8]{inputenc}
\usepackage[english]{babel}
\usepackage{amsmath}
\usepackage{amsfonts}
\usepackage{amssymb}
\begin{document}
\section{Coursera}
\subsubsection{Week 1}
Machine Learning algorithms :
	Supervised learning
	Unsupervised learning
	Reinforcement Learning
	Recommender systems
	
	
Supervised Learning
	based on right answers
	Regression problem: predict continuous valued output (price)
	Classification problem: output value of 0 or 1
	%% COURSERA TEXT
	To describe the supervised learning problem slightly more formally, our goal is, given a training set, to learn a function h : X -> Y so that h(x) is a good predictor for the corresponding value of y. For historical reasons, this function h is called a hypothesis. Seen pictorially, the process is therefore like this:
%%%%
	
Unsupervised Learning
%% COURSERA TEXT
	Unsupervised learning allows us to approach problems with little or no idea what our results should look like. We can derive structure from data where we don't necessarily know the effect of the variables.

We can derive this structure by clustering the data based on relationships among the variables in the data.

With unsupervised learning there is no feedback based on the prediction results.
%%%%
	Not told what to do -> find some structure in data
	Clustering Algorithm
	not give the algorithm the right answers
	Cocktail party problem: take a data that seems dated together and separate them (two audio sources),  
	

Hypothesis Function
	
	
Cost Function
	Cost Function measures the accuracy of our hypothesis function.
		

Gradient descent Algorithm
	Optimize parameters in hypothesis function to minimize the cost function
	Convex functions has only a global optima, no local optima (good for working with)
	"Batch" gradient descent: each step of gradient descent uses all the training examples
	
	
Learning rate
	Controls how big the step is in the optimization process
	
	
\subsubsection{Week 2}
Fix a notation thorughout the work
	m -> the number of training examples i
	n -> the number of features j\\
	$ x^{(i)}_{j} $\\
	$ i = 1,...,m $\\
	$ j = 0,...,n $
	i are the lines and j the columns.
Gradient descent for multiple variables
	
Feature scale
	mean normalize data to converge faster (do not apply to inserted $ x_{0} $ that equals 1 for example). Adjust input values as shown in this formula.
	\begin{equation}
	x_{i} := \frac{x_{i}-\mu_{i}}{s_{i}}
	\end{equation}
	
	where $ \mu_{i} $ is the average of all values for feature (i) and $ s_{i} $ is the range of values ($ max(x_{i})-min(x_{i}) $) or the standard deviation of a given feature $ i $.
	get all features between -1 and 1 range or 0 and 1
	
Gradient descent $\mathcal{O}(n^{2})$
	\begin{equation}
	\theta_{j} := \theta_{j} - \alpha \frac{\partial}{\partial \theta_{j}}J(\theta)
	\end{equation}
	works well when n is large
	
Learning rate
	how to choose: look graph min J vs No. of iterations. It should decrease at every iteration.

Polynomial regression
	Create new features to fit a polynomial regression to a linear regression.
	E.g.
	
	\begin{equation}
	x_{0} = (feature)
	\end{equation}
	\begin{equation}
	x_{1} = (feature)^{2}
	\end{equation}
	
	
Analytic solution (faster convergence)
	The value of $ \theta = (X^{T}X)^{-1}X^{T}y $ (normal equation) gives the optimal value of $\theta$ that minimizes $ J(\theta) $
	Don't need to iterate nor choose $\alpha$.
	Compute the inverse costs a lot $\mathcal{O}(n^{3})$. So it is preferably to use if n is small over the iterative method.
	Noninvertibility: pinv -> pseudo inverse: calculates the value of the inverse even if it is non inversible. 
	
\end{document}