\documentclass[12pt,a4paper]{report}
\usepackage[utf8]{inputenc}
\usepackage[english]{babel}
\usepackage{amsmath}
\usepackage{amsfonts}
\usepackage{amssymb}
\begin{document}
\section{Coursera}
\subsubsection{Week 1}
Machine Learning algorithms :
	Supervised learning
	Unsupervised learning
	Reinforcement Learning
	Recommender systems
	
	
Supervised Learning
	based on right answers
	Regression problem: predict continuous valued output (price)
	Classification problem: output value of 0 or 1
	%% COURSERA TEXT
	To describe the supervised learning problem slightly more formally, our goal is, given a training set, to learn a function h : X → Y so that h(x) is a “good” predictor for the corresponding value of y. For historical reasons, this function h is called a hypothesis. Seen pictorially, the process is therefore like this:
%%%%
	
Unsupervised Learning
%% COURSERA TEXT
	Unsupervised learning allows us to approach problems with little or no idea what our results should look like. We can derive structure from data where we don't necessarily know the effect of the variables.

We can derive this structure by clustering the data based on relationships among the variables in the data.

With unsupervised learning there is no feedback based on the prediction results.
%%%%
	Not told what to do -> find some structure in data
	Clustering Algorithm
	not give the algorithm the right answers
	Cocktail party problem: take a data that seems dated together and separate them (two audio sources),  
	

Hypothesis Function
	
	
Cost Function
	Cost Function measures the accuracy of our hypothesis function.
		


	
\end{document}