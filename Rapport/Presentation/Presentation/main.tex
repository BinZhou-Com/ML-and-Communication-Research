% Copyright 2004 by Till Tantau <tantau@users.sourceforge.net>.
%
% In principle, this file can be redistributed and/or modified under
% the terms of the GNU Public License, version 2.
%
% However, this file is supposed to be a template to be modified
% for your own needs. For this reason, if you use this file as a
% template and not specifically distribute it as part of a another
% package/program, I grant the extra permission to freely copy and
% modify this file as you see fit and even to delete this copyright
% notice. 

\documentclass{beamer}

% There are many different themes available for Beamer. A comprehensive
% list with examples is given here:
% http://deic.uab.es/~iblanes/beamer_gallery/index_by_theme.html
% You can uncomment the themes below if you would like to use a different
% one:
%\usetheme{AnnArbor}
%\usetheme{Antibes}
%\usetheme{Bergen}
%\usetheme{Berkeley}
%\usetheme{Berlin}
%\usetheme{Boadilla}
%\usetheme{boxes}
%\usetheme{CambridgeUS}
%\usetheme{Copenhagen}
%\usetheme{Darmstadt}
%\usetheme{default}
%\usetheme{Frankfurt}
%\usetheme{Goettingen}
%\usetheme{Hannover}
%\usetheme{Ilmenau}
%\usetheme{JuanLesPins}
%\usetheme{Luebeck}
\usetheme{Madrid}
%\usetheme{Malmoe}
%\usetheme{Marburg}
%\usetheme{Montpellier}
%\usetheme{PaloAlto}
%\usetheme{Pittsburgh}
%\usetheme{Rochester}
%\usetheme{Singapore}
%\usetheme{Szeged}
%\usetheme{Warsaw}

\title[Machine Learning and Communication]{Machine Learning Autoencoder Applied to Communication Channels}

\author{E.~Dadalto Camara Gomes\inst{1} \and M.~Benammar\inst{2}}
% - Give the names in the same order as the appear in the paper.
% - Use the \inst{?} command only if the authors have different
%   affiliation.

\institute[] % (optional, but mostly needed)
{
  \inst{1}%
   ISAE-SUPAERO\\
   Université de Toulouse\\
    31055, Toulouse, France\\
Email: eduardo.dadalto-camara-gomes@student.isae-supaero.fr
  \and
  \inst{2}%
  Department of Electronics, Optronics, and Signal processing\\
  ISAE-SUPAERO\\
  31055, Toulouse, France\\
  Email: meryem.benammar@isae-supaero.fr}
% - Use the \inst command only if there are several affiliations.
% - Keep it simple, no one is interested in your street address.

\date{ISAE-SUPAERO, 2019}
% - Either use conference name or its abbreviation.
% - Not really informative to the audience, more for people (including
%   yourself) who are reading the slides online

\subject{Machine Learning and Communication}
% This is only inserted into the PDF information catalog. Can be left
% out. 

% If you have a file called "university-logo-filename.xxx", where xxx
% is a graphic format that can be processed by latex or pdflatex,
% resp., then you can add a logo as follows:

\pgfdeclareimage[height=0.5cm]{university-logo}{images/ISAE}
\logo{\pgfuseimage{university-logo}}

% Delete this, if you do not want the table of contents to pop up at
% the beginning of each subsection:
%\AtBeginSubsection[]
%{
%  \begin{frame}<beamer>{Outline}
%    \tableofcontents[currentsection,currentsubsection]
%  \end{frame}
%}

% Let's get started
\begin{document}

\begin{frame}
  \titlepage
\end{frame}

\begin{frame}{Outline}
  \tableofcontents
  % You might wish to add the option [pausesections]
\end{frame}

% Section and subsections will appear in the presentation overview
% and table of contents.
\section{Introduction}

\begin{frame}{Context}{Communication system context in general - what field will I be treating}
  \begin{itemize}
  \item {
    My first point.
  }
  \item {
    My second point.
  }
  \end{itemize}
\end{frame}


\begin{frame}{Context}{Machine Learning applications - what could we do in communication system}
  \begin{itemize}
  \item {
    My first point.
  }
  \item {
    My second point.
  }
  \end{itemize}
\end{frame}

\begin{frame}{Relevance \& Challenges}{Explain why the work is relevant and explain what are the challenges}
  \begin{itemize}
  \item {
    My first point.
  }
  \item {
    My second point.
  }
  \end{itemize}
\end{frame}

\begin{frame}{Problem Statement}{What exactly I will solve in this work}
  \begin{itemize}
  \item {
    My first point.
  }
  \item {
    My second point.
  }
  \end{itemize}
\end{frame}


% You can reveal the parts of a slide one at a time
% with the \pause command:
\begin{frame}{Second Slide Title}
  \begin{itemize}
  \item {
    First item.
    \pause % The slide will pause after showing the first item
  }
  \item {   
    Second item.
  }
  % You can also specify when the content should appear
  % by using <n->:
  \item<3-> {
    Third item.
  }
  \item<4-> {
    Fourth item.
  }
  % or you can use the \uncover command to reveal general
  % content (not just \items):
  \item<5-> {
    Fifth item. \uncover<6->{Extra text in the fifth item.}
  }
  \end{itemize}
\end{frame}


\section{Methodology}

\subsection{Reference Model}
\begin{frame}{Maximum a Posterior (MAP) Rule}{Implementation of a MAP decoder for a linear block code through a BSC.}
  \begin{itemize}
  \item {
    My first point.
  }
  \item {
    My second point.
  }
  \end{itemize}
\end{frame}

\subsection{Design \& Architecture}
\begin{frame}{Neural Network's Architecture}{Show the architecture used for each case and remarks some important parameters}
  \begin{itemize}
  \item {
    My first point.
  }
  \item {
    My second point.
  }
  \end{itemize}
\end{frame}

\subsection{Training}
\begin{frame}{Neural Network's Training}{Show the best training parameters for each structure}
  \begin{itemize}
  \item {
    My first point.
  }
  \item {
    My second point.
  }
  \end{itemize}
\end{frame}

\subsection{Predictions}
\begin{frame}{Monte Carlo Simulations}{Explain how we could use NN to predict the results with certain confidence.}
  \begin{itemize}
  \item {
    My first point.
  }
  \item {
    My second point.
  }
  \end{itemize}
\end{frame}


\begin{frame}{Blocks}
\begin{block}{Block Title}
You can also highlight sections of your presentation in a block, with it's own title
\end{block}
\begin{theorem}
There are separate environments for theorems, examples, definitions and proofs.
\end{theorem}
\begin{example}
Here is an example of an example block.
\end{example}
\end{frame}

% Placing a * after \section means it will not show in the
% outline or table of contents.

\section{Results \& Discussions}
\subsection{DNN Decoders}
\begin{frame}{DNN Array Decoder}{Show the results for the array decoder in terms of train p, Mep, Parameters, etc}
  \begin{itemize}
  \item {
    My first point.
  }
  \item {
    My second point.
  }
  \end{itemize}
\end{frame}

\begin{frame}{DNN One-hot Decoder}{Show the results for the one-hot decoder in terms of train p, Mep, Parameters, etc}
  \begin{itemize}
  \item {
    My first point.
  }
  \item {
    My second point.
  }
  \end{itemize}
\end{frame}

\subsection{DNN Autoencoders}
\begin{frame}{DNN Autoencoder}{Show the results for the autoencoder in terms of train p, Mep, Parameters, etc}
  \begin{itemize}
  \item {
    My first point.
  }
  \item {
    My second point.
  }
  \end{itemize}
\end{frame}

\section{Conclusions}
\begin{frame}{Conclusions}
  \begin{itemize}
  \item {
    My first point.
  }
  \item {
    My second point.
  }
  \end{itemize}
\end{frame}

\section{Future Work}
\begin{frame}{Future Work}
  \begin{itemize}
  \item {
    My first point.
  }
  \item {
    My second point.
  }
  \end{itemize}
\end{frame}

\section*{Acknowledgments

\begin{frame}{Acknowledgments}
  \begin{itemize}
  \item {
    My first point.
  }
  \item {
    My second point.
  }
  \end{itemize}
\end{frame}

% All of the following is optional and typically not needed. 
\appendix
\section<presentation>*{\appendixname}
\subsection<presentation>*{Bibliography}

\begin{frame}[allowframebreaks]
  \frametitle<presentation>{Bibliography}
    
  \begin{thebibliography}{10}
    
  \beamertemplatebookbibitems
  % Start with overview books.

  \bibitem{Author1990}
    A.~Author.
    \newblock {\em Handbook of Everything}.
    \newblock Some Press, 1990.
 
    
  \beamertemplatearticlebibitems
  % Followed by interesting articles. Keep the list short. 

  \bibitem{Someone2000}
    S.~Someone.
    \newblock On this and that.
    \newblock {\em Journal of This and That}, 2(1):50--100,
    2000.
  \end{thebibliography}
\end{frame}

\end{document}


