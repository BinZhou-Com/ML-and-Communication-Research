\documentclass{beamer}

% You can uncomment the themes below if you would like to use a different
% one:
%\usetheme{AnnArbor}
%\usetheme{Antibes}
%\usetheme{Bergen}
%\usetheme{Berkeley}
%\usetheme{Berlin}
%\usetheme{Boadilla}
%\usetheme{boxes}
%\usetheme{CambridgeUS}
%\usetheme{Copenhagen}
%\usetheme{Darmstadt}
%\usetheme{default}
%\usetheme{Frankfurt}
%\usetheme{Goettingen}
%\usetheme{Hannover}
%\usetheme{Ilmenau}
%\usetheme{JuanLesPins}
%\usetheme{Luebeck}
\usetheme{Madrid}
%\usetheme{Malmoe}
%\usetheme{Marburg}
%\usetheme{Montpellier}
%\usetheme{PaloAlto}
%\usetheme{Pittsburgh}
%\usetheme{Rochester}
%\usetheme{Singapore}
%\usetheme{Szeged}
%\usetheme{Warsaw}

\usepackage{cite}
\usepackage{amsmath,amssymb,amsfonts}

\usepackage{graphicx}
\usepackage{textcomp}
\usepackage{xcolor}
\usepackage{tikz}
\usepackage{verbatim}
\usepackage{algorithm}
\usepackage[noend]{algpseudocode}
\usepackage{multirow}
\usepackage{url}
%\usepackage[sorting=none]{biblatex}
\usepackage[utf8]{inputenc}
\usepackage{ifthen}
\usepackage{filecontents}
\usetikzlibrary{shapes,arrows,shadings,patterns}
\usepackage{pgfplots}
\pgfplotsset{compat=newest}
\pgfplotsset{plot coordinates/math parser=false}

\makeatletter
\def\BState{\State\hskip-\ALG@thistlm}
\makeatother

\usetikzlibrary{positioning}
\def\BibTeX{{\rm B\kern-.05em{\sc i\kern-.025em b}\kern-.08em
    T\kern-.1667em\lower.7ex\hbox{E}\kern-.125emX}}
    
\makeatletter
\def\endthebibliography{%
  \def\@noitemerr{\@latex@warning{Empty `thebibliography' environment}}%
  \endlist
}
\makeatother    
    
\makeatletter
\def\thickhline{%
  \noalign{\ifnum0=`}\fi\hrule \@height \thickarrayrulewidth \futurelet
   \reserved@a\@xthickhline}
\def\@xthickhline{\ifx\reserved@a\thickhline
               \vskip\doublerulesep
               \vskip-\thickarrayrulewidth
             \fi
      \ifnum0=`{\fi}}
\makeatother

\newlength{\thickarrayrulewidth}
\setlength{\thickarrayrulewidth}{3\arrayrulewidth}   


\title[Machine Learning and Communication]{Machine Learning Autoencoder Applied to Communication Channels}

\author{E.~Dadalto Camara Gomes\inst{1} \and M.~Benammar\inst{2}}
% - Give the names in the same order as the appear in the paper.
% - Use the \inst{?} command only if the authors have different
%   affiliation.

\institute[] % (optional, but mostly needed)
{
  \inst{1}%
   ISAE-SUPAERO\\
   Université de Toulouse\\
    31055, Toulouse, France\\
Email: eduardo.dadalto-camara-gomes@student.isae-supaero.fr
  \and
  \inst{2}%
  Department of Electronics, Optronics, and Signal processing\\
  ISAE-SUPAERO\\
  31055, Toulouse, France\\
  Email: meryem.benammar@isae-supaero.fr}
% - Use the \inst command only if there are several affiliations.
% - Keep it simple, no one is interested in your street address.

\date{ISAE-SUPAERO, 2019}
% - Either use conference name or its abbreviation.
% - Not really informative to the audience, more for people (including
%   yourself) who are reading the slides online

\subject{Machine Learning and Communication}
% This is only inserted into the PDF information catalog. Can be left
% out. 

% If you have a file called "university-logo-filename.xxx", where xxx
% is a graphic format that can be processed by latex or pdflatex,
% resp., then you can add a logo as follows:

\pgfdeclareimage[height=0.5cm]{university-logo}{images/ISAE}
\logo{\pgfuseimage{university-logo}}

% Delete this, if you do not want the table of contents to pop up at
% the beginning of each subsection:
%\AtBeginSubsection[]
%{
%  \begin{frame}<beamer>{Outline}
%    \tableofcontents[currentsection,currentsubsection]
%  \end{frame}
%}

% Let's get started
\begin{document}

\begin{frame}
  \titlepage
\end{frame}

\begin{frame}{Outline}
  \tableofcontents
  % You might wish to add the option [pausesections]
\end{frame}

% Section and subsections will appear in the presentation overview
% and table of contents.
\section{Introduction}

\begin{frame}{Context}{Communication system context in general - what field will I be treating}
  \begin{itemize}
  \item {
    My first point.
  }
  \item {
    My second point.
  }
  \end{itemize}
\end{frame}


\begin{frame}{Context}{Machine Learning applications - what could we do in communication system}
  \begin{itemize}
  \item {
    My first point.
  }
  \item {
    My second point.
  }
  \end{itemize}
\end{frame}

\begin{frame}{Relevance \& Challenges}{Explain why the work is relevant and explain what are the challenges}
  \begin{itemize}
  \item {
    My first point.
  }
  \item {
    My second point.
  }
  \end{itemize}
\end{frame}

\begin{frame}{Problem Statement}{What exactly I will solve in this work}
  \begin{itemize}
  \item {
    My first point.
  }
  \item {
    My second point.
  }
  \end{itemize}
\end{frame}


% You can reveal the parts of a slide one at a time
% with the \pause command:
\begin{frame}{Second Slide Title}
  \begin{itemize}
  \item {
    First item.
    \pause % The slide will pause after showing the first item
  }
  \item {   
    Second item.
  }
  % You can also specify when the content should appear
  % by using <n->:
  \item<3-> {
    Third item.
  }
  \item<4-> {
    Fourth item.
  }
  % or you can use the \uncover command to reveal general
  % content (not just \items):
  \item<5-> {
    Fifth item. \uncover<6->{Extra text in the fifth item.}
  }
  \end{itemize}
\end{frame}


\section{Methodology}

\subsection{Reference Model}
\begin{frame}{Maximum a Posterior (MAP) Rule}{Implementation of a MAP decoder for a linear block code through a BSC.}
  \begin{itemize}
  \item {
    My first point.
  }
  \item {
    My second point.
  }
  \end{itemize}
\end{frame}

\subsection{Design \& Architecture}
\begin{frame}{Neural Network's Architecture}{Show the architecture used for each case and remarks some important parameters}
  \begin{itemize}
  \item {
    
  }
  \item {
    
  }
  \item {
    
  }
  \end{itemize}
\end{frame}

\subsection{Training}
\begin{frame}{Neural Network's Training}{Show the best training parameters for each structure}
  \begin{itemize}
  \item {
    My first point.
  }
  \item {
    My second point.
  }
  \end{itemize}
\end{frame}

\subsection{Predictions}
\begin{frame}{Monte Carlo Simulations}{Explain how we could use NN to predict the results with certain confidence.}
  \begin{itemize}
  \item {
    My first point.
  }
  \item {
    My second point.
  }
  \end{itemize}
\end{frame}


\begin{frame}{Blocks}
\begin{block}{Block Title}
You can also highlight sections of your presentation in a block, with it's own title
\end{block}
\begin{theorem}
There are separate environments for theorems, examples, definitions and proofs.
\end{theorem}
\begin{example}
Here is an example of an example block.
\end{example}
\end{frame}

% Placing a * after \section means it will not show in the
% outline or table of contents.

\section{Results \& Discussions}
\subsection{DNN Decoders}
\begin{frame}{DNN Array Decoder}{Show the results for the array decoder in terms of train p, Mep, Parameters, etc}
 
    \begin{figure}[!ht]
  \centering
    \includegraphics[width=0.5\textwidth]{../../Article/images/MLNN_Mep_65536_ptrain_007}
    \caption{Array decoding BER performance. NN trained with a channel crossover probability error of $p_t=0.07$.}\label{fig:ArrayD}
\end{figure}
 
\end{frame}

\begin{frame}{DNN One-hot Decoder}{Show the results for the one-hot decoder in terms of train p, Mep, Parameters, etc}
  
    \begin{figure}[!ht]
  \centering
    \includegraphics[width=0.5\textwidth]{../../Article/images/MLNN1H_Mep_16384_ptrain_0}
    \caption{One hot decoding BER performance. NN decoder trained with a channel crossover probability error of $p_t=0$.}\label{fig:1HD}
\end{figure}

\end{frame}

\subsection{DNN Autoencoders}
\begin{frame}{DNN Autoencoder}{Show the results for the autoencoder in terms of train p, Mep, Parameters, etc}
  \begin{itemize}
  \item {
    My first point.
  }
  \item {
    My second point.
  }
  \end{itemize}
\end{frame}

\section{Conclusions}
\begin{frame}{Conclusions}
  \begin{itemize}
  \item {
    My first point.
  }
  \item {
    My second point.
  }
  \end{itemize}
\end{frame}

\section{Future Work}
\begin{frame}{Future Work}
  \begin{itemize}
  \item {
    My first point.
  }
  \item {
    My second point.
  }
  \end{itemize}
\end{frame}

\section*{Acknowledgment}

\begin{frame}{Acknowledgment}
  \begin{itemize}
  \item {
    My first point.
  }
  \item {
    My second point.
  }
  \end{itemize}
\end{frame}

% All of the following is optional and typically not needed. 
%\appendix
%\section<presentation>*{Bibliography}

%\begin{frame}[allowframebreaks]
%  \frametitle<presentation>{Bibliography}
    

 % \cite{Shannon:2001:MTC:584091.584093, DBLP:journals/corr/CalabreseWGPS16, DBLP:journals/corr/OSheaH17, 2016arXiv160806409O, 2017arXiv171008379G, Worm00turbo-decodingwithout, Viterbi, journals/ett/RobertsonHV97, Ibnkahla, nielsenneural, murphy2013machine, DBLP:journals/corr/AbadiABBCCCDDDG16, chollet2015keras, doi:10.1162/neco.2006.18.7.1527, DBLP:conf/acssc/BenammarP18}    
    
 % \bibliographystyle{ieeetr}
 % \bibliography{../../Article/bib/ShannonCE1948,../../Article/bib/CalabreseWGPS16,../../Article/bib/OSheaH17,../../Article/bib/Oshea2,../../Article/bib/Gruber,../../Article/bib/Worm,../../Article/bib/Viterbi,../../Article/bib/RobertsonHV97,../../Article/bib/Ibnkahla,../../Article/bib/Nielsen,../../Article/bib/Murphy,../../Article/bib/AbadiABBCCCDDDG16,../../Article/bib/Keras,../../Article/bib/mit_neco18_1527,../../Article/bib/BenammarP18}
 
%\end{frame}

\end{document}


